\documentclass[a4paper,10pt,twoside]{article}

\usepackage[top=1in, bottom=1in, left=1in, right=1in]{geometry}
\usepackage[utf8]{inputenc}
\usepackage[spanish,es-ucroman,es-noquoting]{babel}
\usepackage{setspace}
\usepackage{fancyhdr}
\usepackage{lastpage}
\usepackage{amsmath}
\usepackage{amsfonts}
\usepackage{verbatim}
\usepackage{graphicx}
\usepackage{float}
\usepackage{algpseudocode}
\usepackage[toc, page]{appendix}

% Evita que el documento se estire verticalmente para ocupar el espacio vacío
% en cada página.
\raggedbottom

% Deshabilita sangría en la primer línea de un párrafo.
\setlength{\parindent}{0em}

% Separación entre párrafos.
\setlength{\parskip}{0.5em}

% Asigna la traducción de la palabra 'Appendices'.
\renewcommand{\appendixtocname}{Apéndices}
\renewcommand{\appendixpagename}{Apéndices}


%%%%%%%%%% Configuración de Fancyhdr - Inicio %%%%%%%%%%
\pagestyle{fancy}
\thispagestyle{fancy}
\lhead{Trabajo Práctico 1 · Algoritmos y Estructuras de Datos III}
\rhead{Lovisolo · Petaccio · Rossi}
\renewcommand{\footrulewidth}{0.4pt}
\cfoot{\thepage /\pageref{LastPage}}

\fancypagestyle{caratula} {
   \fancyhf{}
   \cfoot{\thepage /\pageref{LastPage}}
   \renewcommand{\headrulewidth}{0pt}
   \renewcommand{\footrulewidth}{0pt}
}
%%%%%%%%%% Configuración de Fancyhdr - Fin %%%%%%%%%%


%%%%%%%%%% Configuración de Algorithmic - Inicio %%%%%%%%%%
% Entorno propio para customizar la presentación del pseudocódigo
\newenvironment{pseudo}[1][]{%
    \vspace{0.5em}%
    \begin{algorithmic}%
}
{%
    \end{algorithmic}%
    \vspace{0.5em}%
}

% Conectivo 'in' para usar así: \ForAll{$foo$ \In $bar$}
\newcommand{\In}{\textbf{in} }

% Complejidades
\newcommand{\Ode}[1]{\hfill $O(#1)$}
%%%%%%%%%% Configuración de Algorithmic - Fin %%%%%%%%%%


\begin{document}


%%%%%%%%%%%%%%%%%%%%%%%%%%%%%%%%%%%%%%%%%%%%%%%%%%%%%%%%%%%%%%%%%%%%%%%%%%%%%%%
%% Carátula                                                                  %%
%%%%%%%%%%%%%%%%%%%%%%%%%%%%%%%%%%%%%%%%%%%%%%%%%%%%%%%%%%%%%%%%%%%%%%%%%%%%%%%


\thispagestyle{caratula}

\begin{center}

\includegraphics[height=2cm]{DC.png} 
\hfill
\includegraphics[height=2cm]{UBA.jpg} 

\vspace{2cm}

Departamento de Computación,\\
Facultad de Ciencias Exactas y Naturales,\\
Universidad de Buenos Aires

\vspace{4cm}

\begin{Huge}
Trabajo Práctico 1
\end{Huge}

\vspace{0.5cm}

\begin{Large}
Algoritmos y Estructuras de Datos III
\end{Large}

\vspace{1cm}

Segundo Cuatrimestre de 2013

\vspace{4cm}

\begin{tabular}{|c|c|c|}
\hline
Apellido y Nombre & LU & E-mail\\
\hline
Leandro Lovisolo      & 645/11 & leandro@leandro.me\\
Lautaro José Petaccio & 443/11 & lausuper@gmail.com\\
Lucas Rossi           & 705/11 & lucasrossi20@gmail.com\\
\hline
\end{tabular}

\end{center}

\newpage


%%%%%%%%%%%%%%%%%%%%%%%%%%%%%%%%%%%%%%%%%%%%%%%%%%%%%%%%%%%%%%%%%%%%%%%%%%%%%%%
%% Índice                                                                    %%
%%%%%%%%%%%%%%%%%%%%%%%%%%%%%%%%%%%%%%%%%%%%%%%%%%%%%%%%%%%%%%%%%%%%%%%%%%%%%%%


\tableofcontents

\newpage


%%%%%%%%%%%%%%%%%%%%%%%%%%%%%%%%%%%%%%%%%%%%%%%%%%%%%%%%%%%%%%%%%%%%%%%%%%%%%%%
%% Introducción                                                              %%
%%%%%%%%%%%%%%%%%%%%%%%%%%%%%%%%%%%%%%%%%%%%%%%%%%%%%%%%%%%%%%%%%%%%%%%%%%%%%%%


\section{Introducción}

En el presente trabajo estudiamos tres problemas algorítmicos, proponemos soluciones para los mismos respetando sus requerimientos de complejidad temporal y analizamos empíricamente los tiempos de ejecución de sus implementaciones en lenguaje C++.

La motivación de este trabajo es comparar las cotas temporales obtenidas del análisis teórico con las mediciones de tiempos de ejecución y extraer conclusiones de esta experimentación.

Sin más, presentamos los problemas estudiados a continuación.


%%%%%%%%%%%%%%%%%%%%%%%%%%%%%%%%%%%%%%%%%%%%%%%%%%%%%%%%%%%%%%%%%%%%%%%%%%%%%%%
%% Problema 1: Pascual y el correo                                           %%
%%%%%%%%%%%%%%%%%%%%%%%%%%%%%%%%%%%%%%%%%%%%%%%%%%%%%%%%%%%%%%%%%%%%%%%%%%%%%%%


\section{Problema 1: Pascual y el correo}

Descripción del problema. Requerimientos de complejidad temporal. Ejemplos de instancias del problema.


\subsection{Solución}

Sea $\langle p_1, \ldots, p_n \rangle$ la secuencia que contiene los pesos de los $n$ paquetes en orden de llegada, sea $\langle c_1, \ldots, c_m \rangle$ la secuencia en orden ascendente que contiene las cargas de los camiones en un determinado momento (inicialmente vacía) y sea $L$ el límite de carga de los camiones. Para cada paquete $p_i$, cargamos el paquete en el camión menos cargado $c_1$ si la nueva carga no supera el límite $L$, o en caso contrario agregamos un nuevo camión $c_{m+1} = p_i$ a la secuencia, realizando las permutaciones necesarias para preservar el orden ascendente. Luego de haber cargado todos los paquetes, devolvemos la secuencia con la carga de cada camión.

Para cumplir con los requerimientos de complejidad temporal, utilizamos una cola de prioridad min-heap en lugar de una secuencia para almacenar las cargas de cada camión.

\begin{pseudo}
    \Procedure{Pascual-y-el-correo}{$\langle p_1, \ldots, p_n \rangle, L$}
        \State $C \leftarrow$ nuevo min-heap                    \Ode{1}
        \ForAll{$p$ \In $\langle p_1, \ldots, p_n \rangle$}     \Ode{1}
            \If{$C.tama\tilde{n}o = 0$}                         \Ode{1}
                \State $C.encolar(p)$                           \Ode{log(n)}
            \ElsIf{$C.siguiente() + p \leq L$}                  \Ode{1}
                \State $c \leftarrow C.siguiente() + p$         \Ode{1}
                \State $C.desencolar()$                         \Ode{log(n)}
                \State $C.encolar(c)$                           \Ode{log(n)}
            \Else
                \State $C.encolar(c)$                           \Ode{log(n)}
            \EndIf
        \EndFor
        \State \textbf{return} $C$
    \EndProcedure
\end{pseudo}


\subsection{Complejidad}

El algoritmo corre en tiempo $O(n*log(n))$, donde $n$ es la cantidad de paquetes.

Demostración: para cada uno de los $n$ paquetes, se realizan 1 o 2 operaciones $O(log(n))$ sobre la cola de prioridad. Luego deducimos la complejidad total $n * O(log(n)) = O(n * log(n))$.


\subsection{Experimentos computacionales}

Pendiente.


%%%%%%%%%%%%%%%%%%%%%%%%%%%%%%%%%%%%%%%%%%%%%%%%%%%%%%%%%%%%%%%%%%%%%%%%%%%%%%%
%% Problema 2: Profesores visitantes                                         %%
%%%%%%%%%%%%%%%%%%%%%%%%%%%%%%%%%%%%%%%%%%%%%%%%%%%%%%%%%%%%%%%%%%%%%%%%%%%%%%%


\section{Problema 2: Profesores visitantes}

Pendiente.


%%%%%%%%%%%%%%%%%%%%%%%%%%%%%%%%%%%%%%%%%%%%%%%%%%%%%%%%%%%%%%%%%%%%%%%%%%%%%%%
%% Problema 3: Una noche en el museo                                         %%
%%%%%%%%%%%%%%%%%%%%%%%%%%%%%%%%%%%%%%%%%%%%%%%%%%%%%%%%%%%%%%%%%%%%%%%%%%%%%%%


\section{Problema 3: Una noche en el museo}

Pendiente.


%%%%%%%%%%%%%%%%%%%%%%%%%%%%%%%%%%%%%%%%%%%%%%%%%%%%%%%%%%%%%%%%%%%%%%%%%%%%%%%
%% Conclusiones                                                              %%
%%%%%%%%%%%%%%%%%%%%%%%%%%%%%%%%%%%%%%%%%%%%%%%%%%%%%%%%%%%%%%%%%%%%%%%%%%%%%%%


\section{Conclusiones}

Pendiente.


%%%%%%%%%%%%%%%%%%%%%%%%%%%%%%%%%%%%%%%%%%%%%%%%%%%%%%%%%%%%%%%%%%%%%%%%%%%%%%%
%% Código fuente para el problema 1                                          %%
%%%%%%%%%%%%%%%%%%%%%%%%%%%%%%%%%%%%%%%%%%%%%%%%%%%%%%%%%%%%%%%%%%%%%%%%%%%%%%%

\newpage

\begin{appendices}

\section{Código fuente para el problema 1}

Pendiente.


%%%%%%%%%%%%%%%%%%%%%%%%%%%%%%%%%%%%%%%%%%%%%%%%%%%%%%%%%%%%%%%%%%%%%%%%%%%%%%%
%% Código fuente para el problema 1                                          %%
%%%%%%%%%%%%%%%%%%%%%%%%%%%%%%%%%%%%%%%%%%%%%%%%%%%%%%%%%%%%%%%%%%%%%%%%%%%%%%%


\section{Código fuente para el problema 2}

Pendiente.


%%%%%%%%%%%%%%%%%%%%%%%%%%%%%%%%%%%%%%%%%%%%%%%%%%%%%%%%%%%%%%%%%%%%%%%%%%%%%%%
%% Código fuente para el problema 3                                          %%
%%%%%%%%%%%%%%%%%%%%%%%%%%%%%%%%%%%%%%%%%%%%%%%%%%%%%%%%%%%%%%%%%%%%%%%%%%%%%%%


\section{Código fuente para el problema 3}

Pendiente.


\end{appendices}

\end{document}